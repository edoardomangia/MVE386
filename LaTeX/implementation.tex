\section{Implementation}
\label{sec:implementation}

This project implements an X-ray simulation workflow that combines a Monte Carlo transport model (Geant4) with a shared, JSON-based scene description. The core goal is to compute voxelized energy deposition in a 3D object for different beam and acquisition configurations, and to derive absorbed dose, temperature rise, and radiolysis products from that deposition. The same scene configuration can also be used for a fast, Beer--Lambert-style gVXR path for quick previews.

\subsection{Scene configuration}
All simulation parameters are defined in a single JSON file (e.g., \texttt{setups/setup.json}). The configuration describes:
\begin{itemize}
  \item \textbf{Beam}: type (parallel or point), source and detector positions, detector pixel grid and pixel size, monoenergetic beam energy, photon flux, and exposure time.
  \item \textbf{Object}: mesh path, units, material formula and density, and heat capacity $c_p$.
  \item \textbf{Voxel grid}: number of voxels and physical half-size of the cubic scoring volume.
  \item \textbf{Acquisition}: rotation mode (step-and-shoot or fly), start/end angles, number of projections, and rotation axis/center.
\end{itemize}
The C++ loader (\texttt{SceneConfig}) resolves relative mesh paths and defines an output directory for each setup.

\subsection{Geant4 simulation pipeline}
The Geant4 executable builds the geometry, defines physics processes, generates primary photons, and accumulates energy deposition:
\begin{itemize}
  \item \textbf{Geometry and materials}: \texttt{DetectorConstruction} loads the STL mesh via CADMesh (Assimp backend) and builds a world volume with air. Materials are created from the chemical formula in the JSON file and assigned a density (g/cm$^3$). A mesh scaling step optionally fits the object into the voxel cube for consistent scoring and visualization.
  \item \textbf{Physics}: \texttt{PhysicsList} registers \texttt{G4EmLivermorePhysics}, which is suitable for low-energy photons. Atomic de-excitation (fluorescence, Auger, PIXE) is enabled to improve realism at keV energies.
  \item \textbf{Primary generation}: \texttt{PrimaryGeneratorAction} emits monoenergetic gamma photons. The beam orientation is controlled by the acquisition schedule: in \emph{fly} mode, the angle is interpolated continuously over all events; in \emph{step} mode, events are grouped into projections. The source, detector, and up-vector are rotated about the configured axis and pivot.
  \item \textbf{Scoring}: \texttt{SteppingAction} records per-step energy deposition (keV) for steps occurring inside the mesh volume. Deposition is accumulated into a voxel grid (\texttt{DoseVoxelGrid}) with thread-safe locking.
  \item \textbf{Output}: \texttt{RunAction} writes a VTI file containing the 3D cell data array (energy deposition in keV per voxel) and metadata such as beam energy, flux, exposure, and event count.
\end{itemize}
The total number of photon events is determined by $N = \Phi \cdot t$, where $\Phi$ is photon flux (photons/s) and $t$ is exposure time (s). For step-and-shoot acquisition, $N$ is multiplied by the number of projections. The run manager optionally chunks events to respect Geant4's 32-bit event limits while keeping a consistent global event index for rotation.

\subsection{What Geant4 is and why it is used}
Geant4 is a Monte Carlo particle transport toolkit widely used in high-energy physics and medical physics. It provides validated interaction models for photons and secondary particles across a wide energy range, as well as a flexible geometry and scoring framework. In this project it is used to model X-ray interactions in complex meshes and to compute spatially resolved energy deposition. That deposition is the physically grounded input for subsequent dose, temperature, and radiolysis calculations.

\subsection{Dose, temperature, and radiolysis post-processing}
After the Geant4 run, Python scripts convert the raw energy deposition to physical quantities:
\begin{itemize}
  \item \textbf{Dose (\texttt{dosage\_geant.py})}: energy deposition in keV is converted to joules and divided by voxel mass to produce absorbed dose in gray (Gy = J/kg). Voxel mass is computed from the voxel volume and material density.
  \item \textbf{Temperature rise (\texttt{heat\_geant.py})}: temperature rise is computed from dose using $\Delta T = D / c_p$, where $c_p$ is the material heat capacity in J/(kg\,K). The output is a VTI file and a NumPy array for analysis.
  \item \textbf{Radiolysis (\texttt{radiolysis\_geant.py})}: for materials with specified G-values (molecules per 100 eV), the script converts deposited energy to molecular yields and then to concentrations (mol/L) and production rates (mol/L/s) per voxel.
\end{itemize}
All post-processing writes both NumPy arrays (for statistics) and VTI files (for ParaView visualization).

\subsection{Theory overview}
The simulation combines fundamental interaction physics with macroscopic dose/heat/radiolysis models:
\begin{itemize}
  \item \textbf{Energy deposition}: X-ray photons undergo photoelectric absorption, Compton scattering, and Rayleigh scattering in the material. Geant4 tracks these interactions and accumulates energy loss locally in voxels.
  \item \textbf{Absorbed dose}: the absorbed dose is the energy deposited per unit mass, $D = E/m$, reported in Gy. This is the primary quantity derived from the Monte Carlo deposition.
  \item \textbf{Temperature rise}: for short exposures and negligible heat diffusion, the temperature increase is proportional to dose, $\Delta T = D / c_p$.
  \item \textbf{Radiolysis}: radiation chemistry yields are modeled using G-values (molecules/100 eV), converting deposited energy into expected molecular concentrations and production rates.
\end{itemize}

\subsection{Batch workflow}
The batch script \texttt{run.sh} compiles the C++ code, runs multiple setups (acquisition modes, beam types, energies, exposures, voxel grids, and materials), and applies the post-processing scripts. For each setup, the outputs include:
\begin{itemize}
  \item \texttt{output/dose\_<setup>.vti} (raw energy deposition)
  \item \texttt{output/dose\_<setup>\_dose\_Gy.vti} and \texttt{.npy} (absorbed dose)
  \item \texttt{output/heat\_<setup>\_deltaT.vti} and \texttt{.npy} (temperature rise)
  \item \texttt{output/rad\_<setup>\_<species>\_M.vti} and \texttt{.npy} (radiolysis products)
\end{itemize}
